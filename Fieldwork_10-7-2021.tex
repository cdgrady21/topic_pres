\documentclass[11pt,ignorenonframetext,xcolor={svgnames}]{beamer}
%\setbeameroption{hide notes}
\usepackage{etex}
\reserveinserts{28}
\usepackage{presentation}
\usepackage{notation}
\usepackage{caption}
\usepackage{amssymb,amsmath}
\usepackage{tikz}
\usepackage{textpos}
\usepackage{ifxetex,ifluatex}
\usepackage{fixltx2e} % provides \textsubscript
%\usepackage{lmodern}
%\usepackage{fontspec,xltxtra,xunicode}
%\defaultfontfeatures{Mapping=tex-text,Scale=MatchLowercase}
%\setmainfont{Lato}
%\setmonofont[Scale=MatchLowercase]{Source Code Pro Light}
%\let\sfdefault\rmdefault
\newcommand{\euro}{€}
\captionsetup{labelformat=empty,labelsep=none,font=small,skip=.25ex} %No "Figure" labels
\usecolortheme{default}
\useinnertheme{default}
\usefonttheme{default}
%\useoutertheme{progressbar}
\usepackage{beamerouterthemeprogressbar}
\progressbaroptions{titlepage=normal,headline=normal,frametitle=normal}


% Define Colors
%% Dark Yellow: 102, 83, 30
%% Dark Orange: 104, 55, 34
%% Dark Red: 107, 36, 31
%% Dark Blue: 19, 64, 80
%% Dark Green: 0, 86, 79
%% yellow:  R=242 G=196 B=70 F2C446
%% orange:  R=249 G=131 B=74 F9834A
%% blue:  R=46 G=154 B=196   2E9AC4
%% red:  R=250 G=86 B=70 FA5646
%% green:  R=3 G=202 B=185 03CAB9
%% grey:  R=153 G=153 B=153 999999
%% Dark Yellow: 102, 83, 30

\definecolor{darkyellow}{rgb}{102,83,30}
\definecolor{orange}{HTML}{F9834A}
\definecolor{yellow}{HTML}{F2C446}
\definecolor{blue}{HTML}{2E9AC4}
\definecolor{darkblue}{rgb}{19,64,80}
\definecolor{green}{HTML}{03CAB9}
\definecolor{grey}{HTML}{999999}
%\definecolor{progressbar@bgblue}{rgb}{0.92, 0.93, 0.95}

%\setbeamerfont*{frametitle}{size=\small,series=\bfseries,bg=white,fg=black}
%\setbeamerfont*{frametitle}{size=\small,series=\bfseries,bg=white,fg=progressbar@bgblue!90!progressbar@fgblue}
\setbeamerfont*{note page}{size=\scriptsize}%%footnotesize}
\setbeamertemplate{note page}[compress]
\setbeamercovered{transparent}
\setbeamertemplate{navigation symbols}{}
\setbeamertemplate{items}[circle]
\setbeamertemplate{sections/subsections in toc}[circle]
\setbeamersize{description width=2ex}
% \setbeamersize{text margin left=1ex,text margin right=1ex}
\setbeamersize{text margin left=2ex,text margin right=1ex}
%%\setbeamersize{ParSkip 1ex plus 1pt minus 1pt}
%% %% %% \setbeamertemplate{caption}[numbered]
\setbeamertemplate{caption label separator}{:}
\setbeamercolor{title}{fg=blue}
\setbeamercolor{author}{fg=grey}
\setbeamercolor{date}{fg=grey}
\setbeamerfont{author}{shape=\itshape}
\setbeamercolor{frametitle}{fg=blue}
\setbeamercolor{caption name}{fg=normal text.fg}

% use upquote if available, for straight quotes in verbatim environments
\IfFileExists{upquote.sty}{\usepackage{upquote}}{}
% use microtype if available
\IfFileExists{microtype.sty}{\usepackage{microtype}}{}
\setbeamerfont{quote}{shape=\upshape}

\setlength{\emergencystretch}{3em}  % prevent overfull lines
\setcounter{secnumdepth}{0}



\usepackage{fancyvrb}


\usepackage{url}

% Comment these out if you don't want a slide with just the
% part/section/subsection/subsubsection title:
%% \AtBeginPart{
%%   \let\insertpartnumber\relax
%%     \let\partname\relax
%%     \frame{\partpage}
%% }

\AtBeginSection{
   \let\insertsectionnumber\relax
      \let\sectionname\relax
  {
    \setbeamercolor{background canvas}{bg=blue}
    %\setbeamercolor{frametitle}{fg=white,bg=darkblue}
    %\setbeamercolor{normal text}{fg=white,bg=darkblue}
    %\setbeamercolor{structure}{fg=white}
       \begin{frame}
    %\frametitle{\sectionpage}
    \tableofcontents[currentsection]
      \end{frame}
  }
}

%%% \AtBeginSubsection{
%%%   \let\insertsubsectionnumber\relax
%%%     \let\subsectionname\relax
%%%     \frame{\subsectionpage}
%%% }
%%%

\setlength{\emergencystretch}{3em}  % prevent overfull lines

\setcounter{secnumdepth}{0}




\title[]{Fieldwork and Field Experiments}


\author{
Chris
Grady\thanks{\href{mailto:cdgrady21@gmail.com}{\nolinkurl{cdgrady21@gmail.com}},
Senior Metrics Advisor, United States Agency for International
Development.} \quad
}

%%\author{true}
%
\date{October 04, 2021}

\graphicspath{{.}{../}} 

%% \addtobeamertemplate{frametitle}{}{%
%% \begin{tikzpicture}[remember picture,overlay]
%% \node[anchor=south east,yshift=6pt,xshift=-3pt] at (current page.south east) {\includegraphics[height=0.5cm]{SBSTLightBulb}};
%% \end{tikzpicture}}
%% 
%% \addtobeamertemplate{title page}{}{%
%% \begin{tikzpicture}[remember picture,overlay]
%% \node[anchor=south east,yshift=5pt] at (current page.south east) {\includegraphics[height=2cm]{SBSTLogo}};
%% \end{tikzpicture}}

\providecommand{\tightlist}{%
  \setlength{\itemsep}{0pt}\setlength{\parskip}{0pt}}

\begin{document}
\begin{frame}[plain,label=intro,noframenumbering]
\titlepage
\end{frame}


%% %% \begin{frame}
%% \tableofcontents[hideallsubsections]
%% \end{frame}
%% 
\begin{frame}
\end{frame}

\begin{frame}{Roadmap}
\protect\hypertarget{roadmap}{}
\begin{itemize}
\tightlist
\item
  Implementation Partners
\item
  Being in the Field
\item
  Intervention and Evaluation

  \begin{itemize}
  \tightlist
  \item
    Data collection methods
  \end{itemize}
\item
  Running the Show
\end{itemize}
\end{frame}

\begin{frame}{What is a field experiment?}
\protect\hypertarget{what-is-a-field-experiment}{}
\begin{itemize}
\tightlist
\item
  An experiment done outside of a lab, with real people who don't know
  they are part of an experiment

  \begin{itemize}
  \tightlist
  \item
    Lab: show people video, survey those people
  \item
    Field: Show a video in a local theatre; survey people who attended
  \end{itemize}
\item
  Researchers assign and administer experimental intervention
\end{itemize}

\smallskip

\begin{itemize}
\tightlist
\item
  Different than natural/quasi-experiment where intervention is
  incidentally random or as-if random
\end{itemize}
\end{frame}

\begin{frame}{Implementation Partners}
\protect\hypertarget{implementation-partners}{}
How to get involved with implementers

\begin{itemize}
\item
  Professor throws you a project
\item
  Meet at a conference
\item
  Pro bono ``cold call'' from you
\item
  University collaborations
\item
  Sites that connect researchers and implementers: research4impact,
  egap, others
\item
  Implementers want prior RCT experience, experience in-country
\item
  Bigger NGOs will want established professors
\end{itemize}
\end{frame}

\begin{frame}{Implementation Partners}
\protect\hypertarget{implementation-partners-1}{}
How to work with non-academics

\begin{itemize}
\tightlist
\item
  Theoretical vs.~practical divide
\item
  Compromise away from perfection
\item
  Avoid jargon
\item
  Explain how long data work can take
\item
  Expect Googledocs
\end{itemize}
\end{frame}

\begin{frame}{Implementation partners}
\protect\hypertarget{implementation-partners-2}{}
Priorities and incentives

\begin{itemize}
\tightlist
\item
  Implementing organizations will want you to do a lot of work that is
  irrelevant for you
\item
  Look our for your own interests
\item
  Implementers \emph{need} to make themselves look good to maintain
  grant funding
\end{itemize}
\end{frame}

\begin{frame}{Being in the Field}
\protect\hypertarget{being-in-the-field}{}
\begin{itemize}
\tightlist
\item
  You need to go
\item
  You want to go
\item
  Establish personal relationships or you will be in the dark and
  nothing will get done
\item
  Bring gifts that are identifiably \emph{American}
\end{itemize}

\smallskip

\begin{itemize}
\tightlist
\item
  Implementer priority: show their project is successful, not to
  evaluate if the project was successful.
\end{itemize}

\note{

You need to be involved with the intervention, not lead it from afar.

Otherwise you'll find out 6 months later that they did not like the randomization and so didn't use it.  Or that they started the intervention before the baseline survey because they didn't want to wait.  Or that a control site and treatment site are less than 1 mile apart.  Or that they changed the intervention and it's totally different now (NGOs are _always_ adapting).

}
\end{frame}

\begin{frame}{Being in the Field}
\protect\hypertarget{being-in-the-field-1}{}
\begin{itemize}
\tightlist
\item
  Learn the language actively
\item
  Learn and follow cultural etiquette
\item
  Calm yourself when you want to start an argument
\end{itemize}
\end{frame}

\begin{frame}{Being in the Field}
\protect\hypertarget{being-in-the-field-2}{}
\begin{itemize}
\tightlist
\item
  Don't bring anything you are afraid to lose
\item
  If not staying in a fancy hotel, be ready for lack of
  water/electricity
\end{itemize}

\medskip

\begin{itemize}
\tightlist
\item
  Start slow and increase intake of local food each day
\item
  Learn what you can and cannot get in-country
\item
  Get a local phone and simcard
\item
  Get personal wifi connection
\end{itemize}
\end{frame}

\begin{frame}{Being in the Field}
\protect\hypertarget{being-in-the-field-3}{}
\begin{itemize}
\tightlist
\item
  Learn to get around without help
\item
  Find amenities and activities -- gyms, basketball/soccer club, local
  bars
\item
  Travel when possible -- do \textbf{not} watch TV in a hotel room
\item
  Hang out with locals
\item
  Bring books to read
\end{itemize}

\medskip

\begin{itemize}
\tightlist
\item
  Enjoy it! Make friends!
\end{itemize}
\end{frame}

\begin{frame}{Interventions and Evaluations}
\protect\hypertarget{interventions-and-evaluations}{}
\begin{itemize}
\tightlist
\item
  Don't let the tail wag the dog -- don't change the intervention so
  that it can be evaluated more easily
\item
  Make implementers part of design discussions so it is a \emph{team}
  design, not \emph{your} design
\item
  Be flexible where things are not imperative
\item
  Be firm where things are necessary
\end{itemize}
\end{frame}

\begin{frame}{Interventions and Evaluations}
\protect\hypertarget{interventions-and-evaluations-1}{}
\begin{itemize}
\tightlist
\item
  Field experiments are a mess -- multiple things will go wrong
\item
  Randomization at multiple levels
\item
  Simple designs are more likely to be implemented faithfully; complex
  designs are beautiful and fragile
\item
  Feasibility extends to survey questions and implementation,
  observational monitoring procedures, everything

  \begin{itemize}
  \tightlist
  \item
    Survey experiments are hard on enumerators because the question
    wording changes every interview
  \end{itemize}
\end{itemize}

\smallskip

\begin{itemize}
\tightlist
\item
  To NGOs, two-arm RCTs are already complicated.\\
\item
  Difficult to pitch more complicated things, like multi-arm designs,
  saturation designs, etc\ldots{}
\end{itemize}
\end{frame}

\begin{frame}{Interventions and Evaluations}
\protect\hypertarget{interventions-and-evaluations-2}{}
\begin{itemize}
\tightlist
\item
  Everything is clustered
\item
  Everything spills over
\end{itemize}

\bigskip

\begin{itemize}
\tightlist
\item
  Get the Gerber and Green book.
\end{itemize}
\end{frame}

\begin{frame}{Data Collection Methods}
\protect\hypertarget{data-collection-methods}{}
\begin{itemize}
\tightlist
\item
  Surveys

  \begin{itemize}
  \tightlist
  \item
    In person
  \item
    Mobile phone ``IVR'' surveys
  \end{itemize}
\item
  Observational measures
\item
  Naturalistic behavioral games
\item
  Focus groups and ``Key Informant'' interviews
\item
  Others?
\end{itemize}

\note{
- Questions in one context make no sense in another context
- Yes/No + how strongly do you feel > likert scales.
- Huge agreement bias.


Would you be comfortable with someone from the same religion but a different denomination marrying into your family?
Would you be comfortable with someone from a different religion marrying into your family?

Would you accept someone from a different religion seeking your hand in marriage without converting? - 85% no
Would you accept someone from a different religion seeking marriage with one of your family members without converting? - 85% no

Would you marry someone you loved, even if they practiced a different religion? - 50% yes, 50% no.


There is more than one valid interpretation of religious teachings.

An Arabic word Bid'ah means innovation, connotation heretical teachings.  Innovations -- unorthodox interpretations of Koranic teachings -- are often blamed for violence in NE Nigeria.  So lots of tolerant people said no.

A woman should have a say in how her household spends money.  

}
\end{frame}

\begin{frame}{Running the Show}
\protect\hypertarget{running-the-show}{}
\begin{itemize}
\tightlist
\item
  You are in charge and there is no reset button or second chance
\item
  Putting your money into it -- sometimes necessary (get per diem)
\item
  Staying up until 3am every night for two weeks
\item
  Hard deadlines
\end{itemize}
\end{frame}

\begin{frame}{Running the Show}
\protect\hypertarget{running-the-show-1}{}
\begin{itemize}
\tightlist
\item
  Lead by example: get sweaty, get cold, get uncomfortable
\item
  Don't ask others to do what you will not do
\item
  Be SUPER ORGANIZED
\end{itemize}
\end{frame}

\begin{frame}{Conclusion}
\protect\hypertarget{conclusion}{}
\begin{itemize}
\tightlist
\item
  Fieldwork and field experiments are great opportunities for
  researchers
\item
  Working with implementation partners can be difficult but rewarding
\item
  Enjoy experiencing another country
\item
  Keep your designs simple
\item
  Use multiple methods of data collection
\item
  Lead by example
\end{itemize}
\end{frame}

\begin{frame}{Conclusion}
\protect\hypertarget{conclusion-1}{}
\begin{itemize}
\tightlist
\item
  Do you really want to do a field experiment/RCT?
\end{itemize}

\medskip

\begin{itemize}
\tightlist
\item
  Take a lot of time
\item
  Require a lot of non-academic work
\item
  High fail rate
\end{itemize}
\end{frame}



\end{document}
